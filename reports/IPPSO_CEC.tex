
%% bare_conf.tex
%% V1.4b
%% 2015/08/26
%% by Michael Shell
%% See:
%% http://www.michaelshell.org/
%% for current contact information.
%%
%% This is a skeleton file demonstrating the use of IEEEtran.cls
%% (requires IEEEtran.cls version 1.8b or later) with an IEEE
%% conference paper.
%%
%% Support sites:
%% http://www.michaelshell.org/tex/ieeetran/
%% http://www.ctan.org/pkg/ieeetran
%% and
%% http://www.ieee.org/

%%*************************************************************************
%% Legal Notice:
%% This code is offered as-is without any warranty either expressed or
%% implied; without even the implied warranty of MERCHANTABILITY or
%% FITNESS FOR A PARTICULAR PURPOSE! 
%% User assumes all risk.
%% In no event shall the IEEE or any contributor to this code be liable for
%% any damages or losses, including, but not limited to, incidental,
%% consequential, or any other damages, resulting from the use or misuse
%% of any information contained here.
%%
%% All comments are the opinions of their respective authors and are not
%% necessarily endorsed by the IEEE.
%%
%% This work is distributed under the LaTeX Project Public License (LPPL)
%% ( http://www.latex-project.org/ ) version 1.3, and may be freely used,
%% distributed and modified. A copy of the LPPL, version 1.3, is included
%% in the base LaTeX documentation of all distributions of LaTeX released
%% 2003/12/01 or later.
%% Retain all contribution notices and credits.
%% ** Modified files should be clearly indicated as such, including  **
%% ** renaming them and changing author support contact information. **
%%*************************************************************************


% *** Authors should verify (and, if needed, correct) their LaTeX system  ***
% *** with the testflow diagnostic prior to trusting their LaTeX platform ***
% *** with production work. The IEEE's font choices and paper sizes can   ***
% *** trigger bugs that do not appear when using other class files.       ***                          ***
% The testflow support page is at:
% http://www.michaelshell.org/tex/testflow/



\documentclass[conference]{IEEEtran}
% Some Computer Society conferences also require the compsoc mode option,
% but others use the standard conference format.
%
% If IEEEtran.cls has not been installed into the LaTeX system files,
% manually specify the path to it like:
% \documentclass[conference]{../sty/IEEEtran}





% Some very useful LaTeX packages include:
% (uncomment the ones you want to load)


% *** MISC UTILITY PACKAGES ***
%
%\usepackage{ifpdf}
% Heiko Oberdiek's ifpdf.sty is very useful if you need conditional
% compilation based on whether the output is pdf or dvi.
% usage:
% \ifpdf
%   % pdf code
% \else
%   % dvi code
% \fi
% The latest version of ifpdf.sty can be obtained from:
% http://www.ctan.org/pkg/ifpdf
% Also, note that IEEEtran.cls V1.7 and later provides a builtin
% \ifCLASSINFOpdf conditional that works the same way.
% When switching from latex to pdflatex and vice-versa, the compiler may
% have to be run twice to clear warning/error messages.






% *** CITATION PACKAGES ***
%
%\usepackage{cite}
% cite.sty was written by Donald Arseneau
% V1.6 and later of IEEEtran pre-defines the format of the cite.sty package
% \cite{} output to follow that of the IEEE. Loading the cite package will
% result in citation numbers being automatically sorted and properly
% "compressed/ranged". e.g., [1], [9], [2], [7], [5], [6] without using
% cite.sty will become [1], [2], [5]--[7], [9] using cite.sty. cite.sty's
% \cite will automatically add leading space, if needed. Use cite.sty's
% noadjust option (cite.sty V3.8 and later) if you want to turn this off
% such as if a citation ever needs to be enclosed in parenthesis.
% cite.sty is already installed on most LaTeX systems. Be sure and use
% version 5.0 (2009-03-20) and later if using hyperref.sty.
% The latest version can be obtained at:
% http://www.ctan.org/pkg/cite
% The documentation is contained in the cite.sty file itself.






% *** GRAPHICS RELATED PACKAGES ***
%
\ifCLASSINFOpdf
  % \usepackage[pdftex]{graphicx}
  % declare the path(s) where your graphic files are
  % \graphicspath{{../pdf/}{../jpeg/}}
  % and their extensions so you won't have to specify these with
  % every instance of \includegraphics
  % \DeclareGraphicsExtensions{.pdf,.jpeg,.png}
\else
  % or other class option (dvipsone, dvipdf, if not using dvips). graphicx
  % will default to the driver specified in the system graphics.cfg if no
  % driver is specified.
  % \usepackage[dvips]{graphicx}
  % declare the path(s) where your graphic files are
  % \graphicspath{{../eps/}}
  % and their extensions so you won't have to specify these with
  % every instance of \includegraphics
  % \DeclareGraphicsExtensions{.eps}
\fi
% graphicx was written by David Carlisle and Sebastian Rahtz. It is
% required if you want graphics, photos, etc. graphicx.sty is already
% installed on most LaTeX systems. The latest version and documentation
% can be obtained at: 
% http://www.ctan.org/pkg/graphicx
% Another good source of documentation is "Using Imported Graphics in
% LaTeX2e" by Keith Reckdahl which can be found at:
% http://www.ctan.org/pkg/epslatex
%
% latex, and pdflatex in dvi mode, support graphics in encapsulated
% postscript (.eps) format. pdflatex in pdf mode supports graphics
% in .pdf, .jpeg, .png and .mps (metapost) formats. Users should ensure
% that all non-photo figures use a vector format (.eps, .pdf, .mps) and
% not a bitmapped formats (.jpeg, .png). The IEEE frowns on bitmapped formats
% which can result in "jaggedy"/blurry rendering of lines and letters as
% well as large increases in file sizes.
%
% You can find documentation about the pdfTeX application at:
% http://www.tug.org/applications/pdftex





% *** MATH PACKAGES ***
%
%\usepackage{amsmath}
% A popular package from the American Mathematical Society that provides
% many useful and powerful commands for dealing with mathematics.
%
% Note that the amsmath package sets \interdisplaylinepenalty to 10000
% thus preventing page breaks from occurring within multiline equations. Use:
%\interdisplaylinepenalty=2500
% after loading amsmath to restore such page breaks as IEEEtran.cls normally
% does. amsmath.sty is already installed on most LaTeX systems. The latest
% version and documentation can be obtained at:
% http://www.ctan.org/pkg/amsmath





% *** SPECIALIZED LIST PACKAGES ***
%
%\usepackage{algorithmic}
% algorithmic.sty was written by Peter Williams and Rogerio Brito.
% This package provides an algorithmic environment fo describing algorithms.
% You can use the algorithmic environment in-text or within a figure
% environment to provide for a floating algorithm. Do NOT use the algorithm
% floating environment provided by algorithm.sty (by the same authors) or
% algorithm2e.sty (by Christophe Fiorio) as the IEEE does not use dedicated
% algorithm float types and packages that provide these will not provide
% correct IEEE style captions. The latest version and documentation of
% algorithmic.sty can be obtained at:
% http://www.ctan.org/pkg/algorithms
% Also of interest may be the (relatively newer and more customizable)
% algorithmicx.sty package by Szasz Janos:
% http://www.ctan.org/pkg/algorithmicx




% *** ALIGNMENT PACKAGES ***
%
%\usepackage{array}
% Frank Mittelbach's and David Carlisle's array.sty patches and improves
% the standard LaTeX2e array and tabular environments to provide better
% appearance and additional user controls. As the default LaTeX2e table
% generation code is lacking to the point of almost being broken with
% respect to the quality of the end results, all users are strongly
% advised to use an enhanced (at the very least that provided by array.sty)
% set of table tools. array.sty is already installed on most systems. The
% latest version and documentation can be obtained at:
% http://www.ctan.org/pkg/array


% IEEEtran contains the IEEEeqnarray family of commands that can be used to
% generate multiline equations as well as matrices, tables, etc., of high
% quality.




% *** SUBFIGURE PACKAGES ***
%\ifCLASSOPTIONcompsoc
%  \usepackage[caption=false,font=normalsize,labelfont=sf,textfont=sf]{subfig}
%\else
%  \usepackage[caption=false,font=footnotesize]{subfig}
%\fi
% subfig.sty, written by Steven Douglas Cochran, is the modern replacement
% for subfigure.sty, the latter of which is no longer maintained and is
% incompatible with some LaTeX packages including fixltx2e. However,
% subfig.sty requires and automatically loads Axel Sommerfeldt's caption.sty
% which will override IEEEtran.cls' handling of captions and this will result
% in non-IEEE style figure/table captions. To prevent this problem, be sure
% and invoke subfig.sty's "caption=false" package option (available since
% subfig.sty version 1.3, 2005/06/28) as this is will preserve IEEEtran.cls
% handling of captions.
% Note that the Computer Society format requires a larger sans serif font
% than the serif footnote size font used in traditional IEEE formatting
% and thus the need to invoke different subfig.sty package options depending
% on whether compsoc mode has been enabled.
%
% The latest version and documentation of subfig.sty can be obtained at:
% http://www.ctan.org/pkg/subfig




% *** FLOAT PACKAGES ***
%
%\usepackage{fixltx2e}
% fixltx2e, the successor to the earlier fix2col.sty, was written by
% Frank Mittelbach and David Carlisle. This package corrects a few problems
% in the LaTeX2e kernel, the most notable of which is that in current
% LaTeX2e releases, the ordering of single and double column floats is not
% guaranteed to be preserved. Thus, an unpatched LaTeX2e can allow a
% single column figure to be placed prior to an earlier double column
% figure.
% Be aware that LaTeX2e kernels dated 2015 and later have fixltx2e.sty's
% corrections already built into the system in which case a warning will
% be issued if an attempt is made to load fixltx2e.sty as it is no longer
% needed.
% The latest version and documentation can be found at:
% http://www.ctan.org/pkg/fixltx2e


%\usepackage{stfloats}
% stfloats.sty was written by Sigitas Tolusis. This package gives LaTeX2e
% the ability to do double column floats at the bottom of the page as well
% as the top. (e.g., "\begin{figure*}[!b]" is not normally possible in
% LaTeX2e). It also provides a command:
%\fnbelowfloat
% to enable the placement of footnotes below bottom floats (the standard
% LaTeX2e kernel puts them above bottom floats). This is an invasive package
% which rewrites many portions of the LaTeX2e float routines. It may not work
% with other packages that modify the LaTeX2e float routines. The latest
% version and documentation can be obtained at:
% http://www.ctan.org/pkg/stfloats
% Do not use the stfloats baselinefloat ability as the IEEE does not allow
% \baselineskip to stretch. Authors submitting work to the IEEE should note
% that the IEEE rarely uses double column equations and that authors should try
% to avoid such use. Do not be tempted to use the cuted.sty or midfloat.sty
% packages (also by Sigitas Tolusis) as the IEEE does not format its papers in
% such ways.
% Do not attempt to use stfloats with fixltx2e as they are incompatible.
% Instead, use Morten Hogholm'a dblfloatfix which combines the features
% of both fixltx2e and stfloats:
%
% \usepackage{dblfloatfix}
% The latest version can be found at:
% http://www.ctan.org/pkg/dblfloatfix




% *** PDF, URL AND HYPERLINK PACKAGES ***
%
%\usepackage{url}
% url.sty was written by Donald Arseneau. It provides better support for
% handling and breaking URLs. url.sty is already installed on most LaTeX
% systems. The latest version and documentation can be obtained at:
% http://www.ctan.org/pkg/url
% Basically, \url{my_url_here}.




% *** Do not adjust lengths that control margins, column widths, etc. ***
% *** Do not use packages that alter fonts (such as pslatex).         ***
% There should be no need to do such things with IEEEtran.cls V1.6 and later.
% (Unless specifically asked to do so by the journal or conference you plan
% to submit to, of course. )


% correct bad hyphenation here
\hyphenation{op-tical net-works semi-conduc-tor}

% *** Algorithm and Algorithmic packages ***
\usepackage{algorithm, algorithmic}


\begin{document}
%
% paper title
% Titles are generally capitalized except for words such as a, an, and, as,
% at, but, by, for, in, nor, of, on, or, the, to and up, which are usually
% not capitalized unless they are the first or last word of the title.
% Linebreaks \\ can be used within to get better formatting as desired.
% Do not put math or special symbols in the title.
\title{IP Based Particle Swarm Optimization\\ for Evolving Deep Convolutional Neural Networks\\ for Image Classification}


% author names and affiliations
% use a multiple column layout for up to three different
% affiliations
\author{
\IEEEauthorblockN{Bin Wang}
\IEEEauthorblockA{School of Engineering\\ and Computer Science\\
	Victoria University of Wellington\\
	Kelburn, Wellington 6012}
\and
\IEEEauthorblockN{Yanan Sun}
\IEEEauthorblockA{School of Engineering\\ and Computer Science\\
	Victoria University of Wellington\\
	Kelburn, Wellington 6012}
\and
\IEEEauthorblockN{Bing Xue}
\IEEEauthorblockA{School of Engineering\\ and Computer Science\\
	Victoria University of Wellington\\
	Kelburn, Wellington 6012}
\and
\IEEEauthorblockN{Mengjie Zhang}
\IEEEauthorblockA{School of Engineering\\ and Computer Science\\
	Victoria University of Wellington\\
	Kelburn, Wellington 6012}
\and
\IEEEauthorblockN{Harith Al-Sahaf}
\IEEEauthorblockA{School of Engineering\\ and Computer Science\\
	Victoria University of Wellington\\
	Kelburn, Wellington 6012}
}


% conference papers do not typically use \thanks and this command
% is locked out in conference mode. If really needed, such as for
% the acknowledgment of grants, issue a \IEEEoverridecommandlockouts
% after \documentclass

% for over three affiliations, or if they all won't fit within the width
% of the page, use this alternative format:
% 
%\author{\IEEEauthorblockN{Michael Shell\IEEEauthorrefmark{1},
%Homer Simpson\IEEEauthorrefmark{2},
%James Kirk\IEEEauthorrefmark{3}, 
%Montgomery Scott\IEEEauthorrefmark{3} and
%Eldon Tyrell\IEEEauthorrefmark{4}}
%\IEEEauthorblockA{\IEEEauthorrefmark{1}School of Electrical and Computer Engineering\\
%Georgia Institute of Technology,
%Atlanta, Georgia 30332--0250\\ Email: see http://www.michaelshell.org/contact.html}
%\IEEEauthorblockA{\IEEEauthorrefmark{2}Twentieth Century Fox, Springfield, USA\\
%Email: homer@thesimpsons.com}
%\IEEEauthorblockA{\IEEEauthorrefmark{3}Starfleet Academy, San Francisco, California 96678-2391\\
%Telephone: (800) 555--1212, Fax: (888) 555--1212}
%\IEEEauthorblockA{\IEEEauthorrefmark{4}Tyrell Inc., 123 Replicant Street, Los Angeles, California 90210--4321}}




% use for special paper notices
%\IEEEspecialpapernotice{(Invited Paper)}




% make the title area
\maketitle

% As a general rule, do not put math, special symbols or citations
% in the abstract
\begin{abstract}
Convolutional neural network (CNN) is one of the most effective deep learning method to solve image classification problems, but the best architecture of CNN to solve a specific problem can be extremely deep which is hardly designed by humans. This paper focuses on utilising Particle Swarm Optimisation (PSO) to automatically search the optimal architecture of CNN without any manual work involved and a novel IP based PSO (IPPSO) - using network IP structure to encode particles, will be proposed. 
\end{abstract}

% no keywords




% For peer review papers, you can put extra information on the cover
% page as needed:
\ifCLASSOPTIONpeerreview
\begin{center} \bfseries EDICS Category: 3-BBND \end{center}
\fi
%
% For peerreview papers, this IEEEtran command inserts a page break and
% creates the second title. It will be ignored for other modes.
\IEEEpeerreviewmaketitle



\section{Introduction}
% no \IEEEPARstart
Convolutional neural network (CNN) has demonstrated its exceptional superiority in numerous machine learning tasks such as speech recognition \cite{CNNspeech:Ossama}, sentence classification \cite{CNNsentence:Yoon} and image classification \cite{ImageNet:Alex}. However, designing CNN for specific tasks can be extremely complex which can be seen from some existing efforts done by researchers such as LeNet \cite{ZipcodeRecognition:LeCun}\cite{DocumentRecognition:LeCun}, AlexNet \cite{ImageNet:Alex}, VGGNet \cite{CNNverydeep:Simonyan} and GoogLeNet \cite{CNNdeeper:Szegedy}. Since the architecture of CNN gets deeper, the hyper-parameters and weights become more complex which makes the further improvement of the CNN architecture harder. In addition, we cannot expect to get the optimal performance by applying the same architecture on various tasks and we need to adjust the CNN architecture for each specific task which will bring tremendous work as there are thousands of types of machine learning tasks in the industry. 


% You must have at least 2 lines in the paragraph with the drop letter
% (should never be an issue)
In order to solve the complex problem of the CNN architecture design, evolutionary computation has recently been leveraged to automatically design the architecture without any human effort involved. Interested researchers have done some excellent work on the automatic design of the CNN architecture by using Genetic Programming \cite{CNNGP:Suganuma} and Genetic Algorithm \cite{CNNevolve:Stanley}. Since PSO is a very simple and effective swarm intelligence algorithm in the evolutionary computation family, in this paper, I would like to develop a PSO method with network IP structure as the encoding scheme to achieve the goal of automatically seeking the optimal CNN architecture. 



\subsection{Goal}
The overall goal of this paper is to design and develop an effective and efficient PSO method - IPPSO to automatically discover good architectures and corresponding weight initialisation values of CNNs. The specific objectives of this paper are to

\begin{enumerate}
	\item Design an IP based particle encoding scheme of CNN architecture, which does not fix the length of the building blocks in CNNs and consists of Convolution , Pooling and Fully-connected layer. With this PSO encoding scheme, the evolved architecture is expected to achieve good performance in solving different tasks at hand. To be specific, each particle will be comprised of a fix-length network interfaces each of which will carry an IP address and its subnet information, and different subnet will define different type of CNN layer. In addition, in order to learn a variable-length CNN architecture, a disabled subnet, that will represent Disabled layers, will be introduced which means the interface having an IP in the disabled subnet will be hidden when decoding the particle.
	\item Based on the designed IPPSO encoding scheme, develop the encoding and decoding methods that convert CNN layer to and IP address and vice versa. 
	\item Design proper subnet ranges for all three types of CNN layers and the disabled layer.  
	\item Develop the associated update equation that can cope with the developed IPPSO encoding strategies of CNN architectures.
	\item Develop a velocity clamping method that are suitable for the novel IPPSO algorithm
	\item Propose an effective fitness measure of the individuals representing different CNNs, which does not require intensive computational resources.
	\item Investigate whether the new approach significantly outperforms the existing methods in classification accuracy
\end{enumerate} 

\subsection{Organisation}
The remaining parts of this paper is organised as follows: first of all, the background of the CNNs, the related works on the architecture design and weight initialization approaches of CNNs are
reviewed in Section II. In addition, the framework and details of each step in the proposed algorithm are elaborated in Section III. Furthermore, the experiment design and experimental results of the proposed algorithm are shown in Sections IV and V, respectively. Next, further discussions are made in Section VI. Last but not least, the conclusions and future work are discussed in Section VII.


\section{Background and related work}


\section{The proposed algorithm}
In this section, the IP based PSO (IPPSO) for evolving deep Convolutional Neural Networks for image classification will be documented in details. 


\subsection{Algorithm Overview}
\begin{algorithm}
	\caption{Framework of IP-PSO}
	\label{alg:framework}
	\begin{algorithmic}
		\renewcommand{\algorithmicrequire}{\textbf{Input:}}
		\renewcommand{\algorithmicensure}{\textbf{Output:}}
		\STATE $P \leftarrow \textit{Initialize the population with the proposed particle}\linebreak \textit{ encoding strategy}$
		\STATE $t \leftarrow 0$
		\STATE $P_{id} \leftarrow empty$
		\STATE $P_{gd} \leftarrow empty$
		\WHILE{$\textit{termination criterion is not satisfied}$}
		\FOR{$\textit{ind in P}$}
		\STATE $ind \leftarrow \textit{update ind using the particle update equation}$
		\IF{$\textit{termination criterion is satisfied}$}
		\STATE $\textbf{break}$
		\ENDIF
		\ENDFOR
		\ENDWHILE		
	\end{algorithmic}
\end{algorithm}

Algorithm \ref{alg:framework} outlines the framework of the proposed algorithm. There are mainly three steps which are really straightforward - initialise the population by using the particle encoding strategy which will be described in Part B, update the position and velocity and check whether the termination criterion meets.

\subsection{Particle Encoding Strategy}
IPPSO encoding strategy is derived from the Network IP addresses. Since CNN is comprised of Convolutional Layer, Pooling Layer, and Fully-Connected Layer and the encoded information of different types of layers varies in terms of both the number of fields and the range in each field shown in Table \ref{table:ConvFields} to Table \ref{table:FullFields}, a fixed length of the Network IP with enough capacity can be designed to accommodate all the types of CNN layers and then the Network IP can be divided into numerous subsets each of which can be used to define one type of CNN layers. 

\begin{table}[!t]
	%% increase table row spacing, adjust to taste
	\renewcommand{\arraystretch}{1.3}
	% if using array.sty, it might be a good idea to tweak the value of
	% \extrarowheight as needed to properly center the text within the cells
	\caption{The fields of Convolutional layer with an example in the Example column}
	\label{table:ConvFields}
	\centering
	%% Some packages, such as MDW tools, offer better commands for making tables
	%% than the plain LaTeX2e tabular which is used here.
	\begin{tabular}{|c|c|c|c|}
		\hline
		Field & Example Value & Range & \# of Bits\\
		\hline
		Filter size & 2(001) & 1-8 & 3\\
		\hline
		\# of feature maps & 32(000 1111) & 1-128 & 7\\
		\hline
		Stride size & 2(01) & 1-4 & 2\\
		\hline
		\textbf{Total} & 001 000 1111 01 &  & 12\\
		\hline
	\end{tabular}
\end{table}


\begin{table}[!t]
	%% increase table row spacing, adjust to taste
	\renewcommand{\arraystretch}{1.3}
	% if using array.sty, it might be a good idea to tweak the value of
	% \extrarowheight as needed to properly center the text within the cells
	\caption{The fields of Pooling layer with an example in the Example column}
	\label{table:PoolingFields}
	\centering
	%% Some packages, such as MDW tools, offer better commands for making tables
	%% than the plain LaTeX2e tabular which is used here.
	\begin{tabular}{|c|c|c|c|}
		\hline
		Field & Example Value & Range & \# of Bits\\
		\hline
		Kernel size & 2(01) & 1-4 & 2\\
		\hline
		Stride size & 2(01) & 1-4 & 2\\
		\hline
		Type:1(maximal), 2(average) & 2(1) & 1-2 & 1\\
		\hline
		Place holder & 32(00 1111) & 1-128 & 6\\
		\hline
		\textbf{Total} & 01 01 0 00 1111 &  & 11\\
		\hline
	\end{tabular}
\end{table}

\begin{table}[!t]
	%% increase table row spacing, adjust to taste
	\renewcommand{\arraystretch}{1.3}
	% if using array.sty, it might be a good idea to tweak the value of
	% \extrarowheight as needed to properly center the text within the cells
	\caption{The fields of Fully-Connected layer with an example in the Example column}
	\label{table:FullFields}
	\centering
	%% Some packages, such as MDW tools, offer better commands for making tables
	%% than the plain LaTeX2e tabular which is used here.
	\begin{tabular}{|c|c|c|c|}
		\hline
		Field & Example Value & Range & \# of Bits\\
		\hline
		\# of Neurons & 1024(011 11111111) & 1-2048 & 11\\
		\hline
		\textbf{Total} & 011 11111111 &  & 11\\
		\hline
	\end{tabular}
\end{table}

\begin{table}[!t]
	%% increase table row spacing, adjust to taste
	\renewcommand{\arraystretch}{1.3}
	% if using array.sty, it might be a good idea to tweak the value of
	% \extrarowheight as needed to properly center the text within the cells
	\caption{Disabled layer with an example in the Example column}
	\label{table:DisabledFields}
	\centering
	%% Some packages, such as MDW tools, offer better commands for making tables
	%% than the plain LaTeX2e tabular which is used here.
	\begin{tabular}{|c|c|c|c|}
		\hline
		Field & Example Value & Range & \# of Bits\\
		\hline
		Place holder & 1024(011 11111111) & 1-2048 & 11\\
		\hline
		\textbf{Total} & 011 11111111 &  & 11\\
		\hline
	\end{tabular}
\end{table}

First of all, the length of the IP liked encoding binary string needs to be designed. As the largest number of bits to represent a layer is 12, there will be 2 bytes required to accommodate the 12 bits IP. 
In addition, the subnets for all types of layers need to be defined and CIDR(Classless Inter-Domain Routing) style will be used to represent the subnet. The subnet 0.0/4 with the range from 0.0 to 15.255 which has the capacity of 12 bits will be used to encode the convolutional layer, the subnet 16.0/5 with the range from 16.0 to 23.255 which has the capacity of 11 bits will represent the fully-connected layer, and the subnet 32.0/5 with the range from 32.0 to 39.255 which has the capacity of 11 bits will carry the information of the Pooling layer. 
Last but not least, as the particle length of PSO is fixed after initialisation,  in order to cope with the variable length of CNN architecture, an alternative way of disabling some of the layers in the encoding vector will be used to accomplish it. Therefore another subnet 32.0/5 with the range from 32.0 to 39.255 will be introduced to mark the layer as not used. 
To summarise, the subnet table where the subsets designed to represent the different types of CNN layers can be drawn as Table \ref{table:Subnets} and each layer will be encoded into a 2 bytes IP address. Table \ref{table:IPExample} shows how the example in Table \ref{table:ConvFields} to Table \ref{table:FullFields} is encoded into IP addresses. 

\begin{table}[!t]
	%% increase table row spacing, adjust to taste
	\renewcommand{\arraystretch}{1.3}
	% if using array.sty, it might be a good idea to tweak the value of
	% \extrarowheight as needed to properly center the text within the cells
	\caption{Four subnets distributed to the three types of CNN layers and the disabled layer}
	\label{table:Subnets}
	\centering
	%% Some packages, such as MDW tools, offer better commands for making tables
	%% than the plain LaTeX2e tabular which is used here.
	\begin{tabular}{|c|c|c|}
		\hline
		Layer type & Subnet(CIDR) & IP Range\\
		\hline
		Convolutional Layer & 0.0/4 & 0.0-15.255\\
		\hline
		Fully-Connected Layer & 16.0/5 & 16.0-23.255\\
		\hline
		Pooling Layer & 24.0/5 & 24.0-31.255\\
		\hline
		Disabled Layer & 32.0/5 & 32.0-39.255\\
		\hline
	\end{tabular}
\end{table}

\begin{table}[!t]
	%% increase table row spacing, adjust to taste
	\renewcommand{\arraystretch}{1.3}
	% if using array.sty, it might be a good idea to tweak the value of
	% \extrarowheight as needed to properly center the text within the cells
	\caption{An example of IP addresses - one for each type of CNN layers}
	\label{table:IPExample}
	\centering
	%% Some packages, such as MDW tools, offer better commands for making tables
	%% than the plain LaTeX2e tabular which is used here.
	\begin{tabular}{|c|c|c|}
		\hline
		Layer type & Binary (filled to 2 bytes) & IP address\\
		\hline
		Convolutional Layer & (0000)001 000 1111 01 & 2.61\\
		\hline
		Pooling Layer & (00000)01 01 0 00 1111 & 18.143\\
		\hline
		Fully-Connected Layer & (00000)011 11111111 & 27.255\\
		\hline
		Disabled Layer & (00000)01111111111 & 35.255\\
		\hline
	\end{tabular}
\end{table}

After converting each layer into a 2 bytes IP address, the position and velocity of PSO can be designed. However, there are a few parameters that need to be defined first - max\_length(maximum length of CNN layers), max\_fully\_connected(maximum fully-connected layers with the constraint of at least one fully-connected layer) listed in Table \ref{table:ParameterList}. And then the encoded data type of the position and the velocity will be a byte array with a fixed length of maximum\_length * 2 and each byte will be deemed as one dimension of the particle.

\begin{table}[!t]
	%% increase table row spacing, adjust to taste
	\renewcommand{\arraystretch}{1.3}
	% if using array.sty, it might be a good idea to tweak the value of
	% \extrarowheight as needed to properly center the text within the cells
	\caption{Parameter list}
	\label{table:ParameterList}
	\centering
	%% Some packages, such as MDW tools, offer better commands for making tables
	%% than the plain LaTeX2e tabular which is used here.
	\begin{tabular}{|p{2.5cm}|p{3cm}|p{2cm}|}
		\hline
		Parameter Name & Parameter Meaning & Value\\
		\hline
		max\_length & maximum length of CNN layers & 9\\
		\hline
		max\_fully\_connected & maximum fully-connected layers given at least there is one fully-connected layer & 3\\
		\hline
		N & population size & 30\\
		\hline
		k & the training epoch number before evaluating the trained CNN & 10\\
		\hline
		num\_of\_batch & the batch size for evaluating the CNN & 200\\
		\hline
		c1 & acceleration coefficient array for $P_{id}$ & 1.49618,1.49618\\
		\hline
		c2 & acceleration coefficient array for $P_{gd}$ & 1.49618,1.49618\\
		\hline
		w & inertia weight for updating velocity & 0.7298\\
		\hline
	\end{tabular}
\end{table}

Here is an example of a particle vector to explain how it can cope with variable-length of CNN architecture. Assume we have the parameter settings - max\_length:5, the particle vector could carry a 5-layer CNN such as $[C, C, P, F, F]$ where C represents Conv layer, P represents Pooling layer, F represents fully-connected layer. After a few PSO updates, the vector could turn the fourth element in the vector from F to D to represents a 4-layer CNN such as $[C, C, P, D, F]$ where D means disabled layer. Therefore, the particle are capable to learn a variable-length CNN architecture with the total number of layers less than 5. 

\subsection{Population Initialisation}

\begin{algorithm}
	\caption{Population Initialisation}
	\label{alg:pop_init}
	\begin{algorithmic}
		\renewcommand{\algorithmicrequire}{\textbf{Input:}}
		\renewcommand{\algorithmicensure}{\textbf{Output:}}
		\REQUIRE $\textit{the population size N,} \newline \textit{the maximum length of CNN layers max\_length,} \newline \textit{the maximum fully-connected layers max\_fully\_connected}$
		\ENSURE $Initialised population P_{0}$
		\STATE $P_{0} \leftarrow \textit{initialise an empty byte array}$
		\WHILE{$\textit{length of }P_{0} < N$}
			\STATE $particle \leftarrow \textit{Initialise a byte array}$
			\IF{$\textit{length of particle}==0$}
				\STATE $L \leftarrow \textit{Initialise a random IP address in the subnet of } \newline \textit{convolutional layer}$
				\STATE $particle \leftarrow \textit{append L at the end of particle byte array}$
			\ENDIF
			\WHILE {$\textit{length of particle} < max\_length - max\_fully\_connected \textbf{ AND } \textit{length of particle} > 0$}
				\STATE $r \leftarrow \textit{Uniformly generate a number between [0, 1]}$
				\IF{$r<0.34$}
					\STATE $L \leftarrow \textit{Initialise a random IP address } \newline \textit{in the subnet of convolutional layer}$
				\ELSIF{$r<0.68$}
					\STATE $L \leftarrow \textit{Initialise a random IP address } \newline \textit{in the subnet of pooling layer}$
				\ELSE
					\STATE $L \leftarrow \textit{Initialise a random IP address } \newline \textit{in the subnet of disabled layer}$
				\ENDIF
				\STATE $particle \leftarrow \textit{append L at the end of particle byte array}$
			\ENDWHILE
			\STATE $is\_fully\_connected \leftarrow \textbf{False}$
			\WHILE {$\textit{length of particle} > max\_length - max\_fully\_connected \textbf{ AND } \textit{length of particle} < max\_length-1$}
				\IF{$is\_fully\_connected == \textbf{True}$}
					\STATE $L \leftarrow \textit{Initialise a random IP address } \newline \textit{in the subnet of fully-connected layer}$
				\ELSE
					\STATE $r \leftarrow \textit{Uniformly generate a number between [0, 1]}$
					\IF{$r<0.25$}
						\STATE $L \leftarrow \textit{Initialise a random IP address } \newline \textit{in the subnet of convolutional layer}$
					\ELSIF{$r<0.5$}
						\STATE $L \leftarrow \textit{Initialise a random IP address } \newline \textit{in the subnet of pooling layer}$
					\ELSIF{$r<0.75$}
						\STATE $L \leftarrow \textit{Initialise a random IP address } \newline \textit{in the subnet of fully-connected layer}$
						\STATE $is\_fully\_connected \leftarrow \textbf{True}$
					\ELSE
						\STATE $L \leftarrow \textit{Initialise a random IP address } \newline \textit{in the subnet of disabled layer}$
					\ENDIF
				\ENDIF
				\STATE $particle \leftarrow \textit{append L at the end of particle byte array}$
			\ENDWHILE
			\STATE $L \leftarrow \textit{Initialise a random IP address in the subnet of } \newline \textit{fully-connected layer}$
			\STATE $particle \leftarrow \textit{append L at the end of particle byte array}$
			\STATE $P_{0} \leftarrow \textit{append particle at the end of }P_{0}$
		\ENDWHILE
		\RETURN $P_{0}$
	\end{algorithmic}
\end{algorithm}

In terms of the population initialisation, as shown in Algorithm \ref{alg:pop_init} we set up the size of the population and randomly create individuals until reaching the population size. 
For each individual, first we initialise an empty vector and each element in it will be used to store a Network Interface containing the IP address and subnet information. The first element will always be a convolutional layer; From the second to (max\_length-max\_fully\_connected) layer each element can be filled with convolutional layer, pooling layer or disabled layer; From (max\_length-max\_fully\_connected) to (max\_length-1) layer it can be filled with any of the four types of layers; And the last element will always be a fully-connected layer. In addition, each layer will be generated with the random settings, a.k.a a random IP address in a specific subnet.


\subsection{Fitness Evaluation}
\begin{algorithm}
	\caption{Fitness Evaluation}
	\label{alg:fitness}
	\begin{algorithmic}
		\renewcommand{\algorithmicrequire}{\textbf{Input:}}
		\renewcommand{\algorithmicensure}{\textbf{Output:}}
		\REQUIRE $\textit{The population } P_{t}, \newline \textit{the training epoch number k,} \newline \textit{the training set } D_{train}, \newline \textit{the fitness evaluation dataset } D_{fitness}, \newline \textit{the batch size } batch\_size$
		\ENSURE $The population with fitness P_{t}$
		\FOR{$individual s \textbf{ in } P_{t}$}
			\STATE $i \leftarrow 1$
			\STATE $eval\_steps \leftarrow \textit{size of } D_{fitness} / batch\_size$
			\WHILE{$i<=k$}
				\STATE $\textit{Train the connection weights of the CNN } \newline \textit{represented by individual s}$
				\IF{$i==k$}
					\STATE $accy\_list \leftarrow \textit{empty array}$
					\WHILE{$j<eval\_steps$}
						\STATE $acc\_j \leftarrow \textit{Evaluate the classification error on } \newline \textit{the j-th batch data from } D_{fitness}$
						\STATE $acc\_list \leftarrow \textit{append accy\_j at the end of accy\_list}$
					\ENDWHILE
					\STATE $mean \leftarrow \textit{calculate the mean value }\newline \textit{of  acc\_list}$
					\STATE $stddev \leftarrow \textit{calculate the standard deviation }\newline \textit{of acc\_list}$
					\STATE $num\_of\_connections \leftarrow \textit{calculate the number }\newline \textit{of connections in s}$
					\STATE $s.fitness \leftarrow (mean, -stddev, -num\_of\_connections)$
				\ENDIF
			\ENDWHILE
		\ENDFOR	
		\RETURN $P_{t}$	
	\end{algorithmic}
\end{algorithm}

With regard to the fitness evaluation in Algorithm \ref{alg:fitness}, each individual is decoded to a CNN with its settings which will be trained for k epoch on the training dataset, and then the trained CNN will be batch-evaluated on the validation dataset which will produce a set of accuracies. Finally, we calculate the mean and standard deviation of the accuracies for each individual which will be stored as the individual fitness along with the number of connections of the CNN. 
For comparing the fitness of individuals, the mean value, standard deviation and the number of parameters will be used in order for the comparison, ie. compare mean value first, if mean value is equal compare standard deviation, if standard deviation is the same compare the number of parameters.


\subsection{Particle Update Equation with Velocity Clamping}
\begin{algorithm}
	\caption{Particle Update Equation with Velocity Clamping}
	\label{alg:update}
	\begin{algorithmic}
		\renewcommand{\algorithmicrequire}{\textbf{Input:}}
		\renewcommand{\algorithmicensure}{\textbf{Output:}}
		\REQUIRE $\textit{particle individual vector ind}, \newline \textit{acceleration coefficient array for $P_{id}$ c1,} \newline \textit{acceleration coefficient array for $P_{gd}$ c2}, \newline \textit{inertia weight w}, \newline \textit{max velocity array } v_{max}$
		\ENSURE $\textit{updated individual vector ind}$
		\FOR{$\textit{element } interface \textbf{ in } ind$}
			\STATE $i=0$
			\FOR{$i < number of bytes of IP address in interface$}
				\STATE $x \leftarrow \textit{the ith byte of the IP address in the interface}$
				\STATE $r1 \leftarrow \textit{uniformly generate r1 between [0, 1]}$
				\STATE $r2 \leftarrow \textit{uniformly generate r2 between [0, 1]}$
				\STATE $v_{new} \leftarrow w * v + c1[i] * r1 * (P_{id} - x) + c2[i] * r2 * (P_{gd} - x)$
				\IF{$v_{max}[i] < v_{new}$}
					\STATE $v_{new} \leftarrow v_{max}[i]$
				\ENDIF
				\STATE $x_{new} \leftarrow x + v_{new}$
				\IF{$x_{new} > 255$}
					\STATE $x_{new} \leftarrow x_{new}-255$
				\ENDIF
			\ENDFOR
		\ENDFOR
		\STATE $fitness \leftarrow \textit{evaluate the updated individual ind}$
		\IF{$P_{id} < fitness$}
			\STATE $P_{id} \leftarrow fitness$
		\ENDIF
		\IF{$P_{gd} < fitness$}
		\STATE $P_{gd} \leftarrow fitness$
		\ENDIF
		\RETURN $ind$
	\end{algorithmic}
\end{algorithm}

In Algorithm \ref{alg:update}, as each layer is encoded into a unit with 2 dimensions in the particle vector and we want to control the acceleration coefficients for each dimension, the two acceleration coefficients are two float arrays with the length of 2. 
After the coefficients defined, we go through each dimension in the individual and update the velocity and position by using the corresponding coefficients for that dimension. Since there are some constraints for each position of the particle vector, e.g. the second element can only be convolutional layer, pooling layer or disabled layer, we need to upgrade the new position by an interface with a random IP address in a valid subnet if the new position does not fall in a valid subnet. And then we evaluate the new position, compare it with its local best and the global best, and then update the two bests if needed.


\subsection{Best Individual Selection and Decoding}

Global best of PSO will be reported as the best individual. In terms of the decoding, we can first identify the type of the layer represented by the network interface - stored in every 2 bytes/dimensions from left to right in the particle vector of the global best, according to the subnets in Table \ref{table:Subnets} we can distinguish the type of layer, and then based on Table \ref{table:ConvFields} to Table \ref{table:FullFields} we can decode the IP into different sets of bits which indicate different fields of the layer. After decoding all the interfaces in the global best, the final CNN architecture can be obtained by connecting all of the decoded layers in the same order as the interfaces in the particle vector. 

\section{Experiment design}


\section{Experimental results and analysis}


\section{Further discussion}



% An example of a floating figure using the graphicx package.
% Note that \label must occur AFTER (or within) \caption.
% For figures, \caption should occur after the \includegraphics.
% Note that IEEEtran v1.7 and later has special internal code that
% is designed to preserve the operation of \label within \caption
% even when the captionsoff option is in effect. However, because
% of issues like this, it may be the safest practice to put all your
% \label just after \caption rather than within \caption{}.
%
% Reminder: the "draftcls" or "draftclsnofoot", not "draft", class
% option should be used if it is desired that the figures are to be
% displayed while in draft mode.
%
%\begin{figure}[!t]
%\centering
%\includegraphics[width=2.5in]{myfigure}
% where an .eps filename suffix will be assumed under latex, 
% and a .pdf suffix will be assumed for pdflatex; or what has been declared
% via \DeclareGraphicsExtensions.
%\caption{Simulation results for the network.}
%\label{fig_sim}
%\end{figure}

% Note that the IEEE typically puts floats only at the top, even when this
% results in a large percentage of a column being occupied by floats.


% An example of a double column floating figure using two subfigures.
% (The subfig.sty package must be loaded for this to work.)
% The subfigure \label commands are set within each subfloat command,
% and the \label for the overall figure must come after \caption.
% \hfil is used as a separator to get equal spacing.
% Watch out that the combined width of all the subfigures on a 
% line do not exceed the text width or a line break will occur.
%
%\begin{figure*}[!t]
%\centering
%\subfloat[Case I]{\includegraphics[width=2.5in]{box}%
%\label{fig_first_case}}
%\hfil
%\subfloat[Case II]{\includegraphics[width=2.5in]{box}%
%\label{fig_second_case}}
%\caption{Simulation results for the network.}
%\label{fig_sim}
%\end{figure*}
%
% Note that often IEEE papers with subfigures do not employ subfigure
% captions (using the optional argument to \subfloat[]), but instead will
% reference/describe all of them (a), (b), etc., within the main caption.
% Be aware that for subfig.sty to generate the (a), (b), etc., subfigure
% labels, the optional argument to \subfloat must be present. If a
% subcaption is not desired, just leave its contents blank,
% e.g., \subfloat[].


% An example of a floating table. Note that, for IEEE style tables, the
% \caption command should come BEFORE the table and, given that table
% captions serve much like titles, are usually capitalized except for words
% such as a, an, and, as, at, but, by, for, in, nor, of, on, or, the, to
% and up, which are usually not capitalized unless they are the first or
% last word of the caption. Table text will default to \footnotesize as
% the IEEE normally uses this smaller font for tables.
% The \label must come after \caption as always.
%
%\begin{table}[!t]
%% increase table row spacing, adjust to taste
%\renewcommand{\arraystretch}{1.3}
% if using array.sty, it might be a good idea to tweak the value of
% \extrarowheight as needed to properly center the text within the cells
%\caption{An Example of a Table}
%\label{table_example}
%\centering
%% Some packages, such as MDW tools, offer better commands for making tables
%% than the plain LaTeX2e tabular which is used here.
%\begin{tabular}{|c||c|}
%\hline
%One & Two\\
%\hline
%Three & Four\\
%\hline
%\end{tabular}
%\end{table}


% Note that the IEEE does not put floats in the very first column
% - or typically anywhere on the first page for that matter. Also,
% in-text middle ("here") positioning is typically not used, but it
% is allowed and encouraged for Computer Society conferences (but
% not Computer Society journals). Most IEEE journals/conferences use
% top floats exclusively. 
% Note that, LaTeX2e, unlike IEEE journals/conferences, places
% footnotes above bottom floats. This can be corrected via the
% \fnbelowfloat command of the stfloats package.




\section{Conclusion and future work}




% conference papers do not normally have an appendix


% use section* for acknowledgment
\section*{Acknowledgement}


The authors would like to thank...





% trigger a \newpage just before the given reference
% number - used to balance the columns on the last page
% adjust value as needed - may need to be readjusted if
% the document is modified later
%\IEEEtriggeratref{8}
% The "triggered" command can be changed if desired:
%\IEEEtriggercmd{\enlargethispage{-5in}}

% references section

% can use a bibliography generated by BibTeX as a .bbl file
% BibTeX documentation can be easily obtained at:
% http://mirror.ctan.org/biblio/bibtex/contrib/doc/
% The IEEEtran BibTeX style support page is at:
% http://www.michaelshell.org/tex/ieeetran/bibtex/
%\bibliographystyle{IEEEtran}
% argument is your BibTeX string definitions and bibliography database(s)
%\bibliography{IEEEabrv,../bib/paper}
%
% <OR> manually copy in the resultant .bbl file
% set second argument of \begin to the number of references
% (used to reserve space for the reference number labels box)
\begin{thebibliography}{11}

\bibitem{CNNspeech:Ossama}
Ossama Abdel-Hamid, Li Deng and Dong Yu, \emph{Exploring Convolutional Neural Network Structures and Optimization Techniques for Speech Recognition}. INTERSPEECH 2013, 5 - 29 August 2013, Lyon, France

\bibitem{CNNsentence:Yoon}
Yoon Kim, \emph{Convolutional Neural Networks for Sentence Classification}, arXiv:1408.5882v2 [cs.CL] 3 Sep 2014

\bibitem{ImageNet:Alex}
Alex Krizhevsky, Ilya Sutskever and Geoffrey E. Hinton, \emph{ImageNet Classification with Deep Convolutional Neural Networks}, Advances in Neural Information Processing Systems 25 (NIPS 2012), 2012

\bibitem{ZipcodeRecognition:LeCun}
Y. LeCun, B. Boser, J. S. Denker, D. Henderson, R. E. Howard, W. Hubbard, and L. D. Jackel, \emph{Backpropagation applied to handwritten zip code recognition}, Neural Computation, vol. 1, no. 4, pp. 541–551, 1989. 

\bibitem{DocumentRecognition:LeCun}
Y. LeCun, L. Bottou, Y. Bengio, and P. Haffner, \emph{Gradient-based learning applied to document recognition}, Proceedings of the IEEE, vol. 86, no. 11, pp. 2278–2324, 1998.

\bibitem{CNNverydeep:Simonyan}
K. Simonyan and A. Zisserman, \emph{Very deep convolutional networks for large-scale image recognition}, arXiv preprint arXiv:1409.1556, 2014.

\bibitem{CNNdeeper:Szegedy}
C. Szegedy, W. Liu, Y. Jia, P. Sermanet, S. Reed, D. Anguelov, D. Erhan, V. Vanhoucke, and A. Rabinovich, \emph{Going deeper with convolutions,” in Proceedings of the IEEE Conference on Computer Vision and Pattern Recognition}, 2015, pp. 1–9.

\bibitem{CNNGP:Suganuma}
Masanori Suganuma, Shinichi Shirakawa, and Tomoharu Nagao, \emph{A Genetic Programming Approach to Designing Convolutional Neural Network Architectures∗}, arXiv:1704.00764v2  [cs.NE]  11 Aug 2017

\bibitem{CNNevolve:Stanley}
K. O. Stanley and R. Miikkulainen, \emph{Evolving neural networks through augmenting topologies}, Evolutionary computation, vol. 10, no. 2, pp. 99–127, 2002.

\bibitem{FashionImage:Xiao}
H. Xiao, K. Rasul, and R. Vollgraf, \emph{Fashion-mnist: a novel image dataset for benchmarking machine learning algorithms}, arXiv preprint arXiv:1708.07747, 2017.

\bibitem{DeepArchitectureEval:Larochelle}
H. Larochelle, D. Erhan, A. Courville, J. Bergstra, and Y. Bengio, \emph{An empirical evaluation of deep architectures on problems with many factors of variation}, in Proceedings of the 24th International Conference on Machine Learning. ACM, 2007, pp. 473–480.

\end{thebibliography}




% that's all folks
\end{document}


